\documentclass[11pt,letterpaper]{article}

%%%%%%%%%%%%%%%%%%%%%%%%%%%%%%%%%%%%%%%%%%%%%%%%%%%%%%%%%%%%%%%%%%%%%%%%%
\pagestyle{plain}                                                      %%
%%%%%%%%%% EXACT 1in MARGINS %%%%%%%                                   %%
\setlength{\textwidth}{6.5in}     %%                                   %%
\setlength{\oddsidemargin}{0in}   %% (It is recommended that you       %%
\setlength{\evensidemargin}{0in}  %%  not change these parameters,     %%
\setlength{\textheight}{8.5in}    %%  at the risk of having your       %%
\setlength{\topmargin}{0in}       %%  proposal dismissed on the basis  %%
\setlength{\headheight}{0in}      %%  of incorrect formatting!!!)      %%
\setlength{\headsep}{0in}         %%                                   %%
\setlength{\footskip}{.5in}       %%                                   %%
%%%%%%%%%%%%%%%%%%%%%%%%%%%%%%%%%%%%                                   %%
\newcommand{\required}[1]{\section*{\hfil #1\hfil}}                    %%


\usepackage{titlesec}
\usepackage{amsmath}
\usepackage{amssymb}
\usepackage{graphicx}
\usepackage[rightcaption]{sidecap}
\usepackage{dsfont}
\usepackage{nopageno}

\usepackage{hyperref}
\usepackage{times}
\usepackage{graphicx} % more modern
%\usepackage{epsfig} % less modern
\usepackage{subfigure} 
\usepackage{url}
\usepackage{epstopdf}
\usepackage{amsfonts}
%\usepackage{caption}
\usepackage{xcolor}
\usepackage{amsmath,amssymb} % define this before the lin numbering.
%\usepackage{ruler}
%\usepackage{subfigure}
% \usepackage{color}
\usepackage{bm}

\renewcommand{\vec}[1]{{\boldsymbol #1}}
\setcounter{secnumdepth}{0}
\titleformat{\section}{\normalfont\Large\bfseries}{\thesection}{1em}{}
\titleformat{\subsection}[display]{\normalfont\bfseries\large}{}{}{}
\titlespacing{\section}{0pt}{13pt}{6.5pt}
\titlespacing{\subsection}{0pt}{11pt}{5.5pt}

\let\oldbibliography\thebibliography
\renewcommand{\thebibliography}[1]{%
  \oldbibliography{#1}%
  \setlength{\itemsep}{0pt}%
}

\begin{document}

\begin{center}

{\Large \textbf{Access Consciousness and Actionable Representations in State-Denoised Recurrent Neural Networks}}

\vspace{1em}
Michael C. Mozer \& Denis Kazakov \\
Department of Computer Science \\
University of Colorado \\
Boulder, CO 80309-0430 USA \\
{\tt \{mozer,kazakov\}@colorado.edu}
\end{center}


\begin{abstract}
HERE IS MY ABSTRACT
\end{abstract}

%\vspace{-.02in} % mozer added to squeeze paragraph onto page

\section{Introduction}

Varieties of consciousness:  access and phenomenal
degrees of consciousness

Global workspace (Dehaene 2017 article)

Conscious states are interpretations 
(Dehaene 
31. T. I. Panagiotaropoulos, G. Deco, V. Kapoor, N. K. Logothetis,
Neuron 74, 924–935 (2012).
32. N. K. Logothetis, Philos. Trans. R. Soc. Lond. B Biol. Sci. 353,
1801–1818 (1998).

stability -- attractor states

 color language terms influence color judgments, a result one would not expect unless (a) conscious states are interpretations, and (b) we can't directly access the low-level perceptual information for decision making.

Effects of language on color discriminability
Witthoft, Winawer, Wu, Frank, Wade, \& Boroditsky (2003)

Relation to Bengio Consciousness prior
- proposal for a simplified or reduced description of an agent's complex, 
internal state that selectis a small subset of the information in the internal 
state, characterized in terms of factors, variables, concepts, dimensions
- emphasis on low dimensionality (e.g., key-value representation where only
one key is selected)
- instead, i focus on coherence of state such that all the bits fit together,
  but may involve multiple factors (e.g., dog -- hair, size, sound, etc.)
- bengio focus is on isolating factors; my goal is to build sharper internal
representations
- bengio's hope is to have a conscious state that can be transformed into
natural language sentences in a fairly simple manner.  in my case, it's more
of a criterion for what makes the conscious state.

Consciousness as a posterior: prior is bias to familiar states, interpretation
is posterior given evidence

Consider any complex intelligent system that is composed of processing components or pathways, e.g., brain, deep net, recurrent net

%\begin{figure}[tb]
%\begin{center}
%\includegraphics[scale=0.45]{128_hand_selected-3.eps}
%\end{center}
%\caption{Three examples showing neural net reconstructions of an original image (center) by the standard approach (left) and by our technique (right). The compression factor is high to highlight differences.}
%\label{fig:example}
%\end{figure}

%\cite{Vincentetal2008}
%\section{Background and related work}
%\subsection{Neural networks for image synthesis}

%\begin{SCfigure}
%\includegraphics[width=2in]{128_autoencoder_ranking_results.png}
%\includegraphics[width=2in]{vae_ranking_results.png}
%\caption{Human judgments of images reconstructed by (a) deterministic and (b) probabilistic convolutional autoencoders. Each graph shows the distribution of image quality ranking for MS-SSIM, MSE, and MAE for 1000 images from the STL-10 hold-out set.}
%\label{fig:human_rankings}
%\end{SCfigure}


%\begin{figure}[bt]
%\centering
%\begin{tabular}{cc}
%(a)
%\includegraphics[width=.35\linewidth]{det_autoencoder_r1_examples.png} &  
%(b)
%\includegraphics[width=.35\linewidth]{vae_recon_pref.png} 
%\end{tabular}
%\caption{Reconstructions from (a) deterministic and (b) probabilistic
%autoencoders. For each architecture, four randomly selected, held-out
%STL-10 images are shown (1st column), along with the MS-SSIM, MSE, and
%MAE reconstructions (2nd-4th columns, respectively).
%}
%\label{fig:128_autoencoder_rank_examples}
%\end{figure}

%\begin{table}[b]
%\small
%\centering
%\begin{tabular}[b]{| c || c | c | c |}
%\hline
%\textbf{Loss Function} &
%\textbf{Azimuth Error} &
%\textbf{Elevation Error} &
%\textbf{Classification Accuracy} \\
%\hline
%\textbf{MS-SSIM} & 514 & 109 & 97.80\% \\
%\textbf{MSE} & 904 & 127 & 95.57\% \\
%\textbf{MAE} & 1046 & 122 &  97.43\% \\
%\hline
%\end{tabular}
%\caption{Regression and Classification Performance on Yale B Face Data Set}
%\label{table:yale}
%\end{table}

\bibliographystyle{apa}
\small
\bibliography{bibliography}

\end{document}
